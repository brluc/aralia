\chapter*{Préface}

Ce petit répertoire est né d'une visite
de terrain avec Daniel Gagnon. Ce biologiste
présentait les aspects concernant la bio-diversité
d'un projet de recherche sur les peupliers commencé
vers 2010 en collaboration avec Ms. Benoit Truax
et Vincent\ldots
Alors que nous marchions dans
une des partie de la foret qui est la plus ancienne,
Daniel pointa le sol dans un endroit qui me semblais
totalement dépourvue d'intérêt et identifia trois plantes.
Il n'en fallait pas plus pour comprendre que
la connaissance de notre milieu est bien pauvre.
Ce même automne nous avions la joie de découvrir
le nom des deux dernières fougères de l'année,
la Matteuccie et la Polystic faux-acrostic.
Cette fin de saison nous laissait aussi avec
une première intrigue, une petite plante d'une seule feuille ressemblant à une feuille de rhubarbe. 
	
Collaboration:
P. C. Morissette,
Fr. L. Lamontagne.
	
Collaboration spéciale pour la programmation
de la base de donné et LaTex:
M. T. Szylowiec.
