\chapter*{Préface}

Ce petit répertoire est né d'une visite
de terrain que fit Daniel Gagnon, doyen de la Faculté des Sciences à l’Université de Regina en Saskatchewan, alors qu'il
présentait les aspects de la bio-diversité
d'un projet de recherche sur les peupliers commencé
vers 2010 en collaboration avec MM. Benoit Truax
et Julien Fortier.
Comme nous marchions dans la partie la plus ancienne de la forêt,
Daniel pointa le sol en un endroit qui semblait
totalement dépourvu d'intérêt et identifia trois plantes.
Il n'en fallait pas plus.
Ce même automne nous avions la joie de découvrir
le nom des deux dernières fougères de l'année,
la Matteuccie et le Polystic faux-acrostic.
Cette fin de saison nous laissait aussi avec
une première intrigue, une petite plante d'une seule feuille ressemblant à celle feuille de la rhubarbe. \\
	
Auteurs:\\
P. C. Morissette,
Fr. L. Lamontagne.
	
Collaboration spéciale:\\  
M. T. Szylowiec.
