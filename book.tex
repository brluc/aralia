\input aralia-preamble

\begin{document}

\begin{titlingpage}
\maketitle
\end{titlingpage}

\chapter*{A}


\herbe{Achillée millefeuille}{Achillea Millefolium}{Composées}{592}

\lipsum[66]

\herbskip

\herbe{Actée blanche}{Actea pachypoda}{Renonculacées}{232}			

Fruits blancs sur pédicelles épais (pachypoda) rouges.
Revers de la feuille moins poilu que \emph{A. rubra.}

\herbskip

\herbe{Actée rouge}{Actea rubra}{}{233}

\lipsum[75]

\herbskip

\herbe{Adiante du Canada}{Adiantum pedatum}{Polypodiacées}{124}

\lipsum[66]

\herbskip
				
\herbe{Agrostide blanche}{Agrostis stolonifera}{Graminées}{793}

Taille: 50 cm. Inflorescence conique rougeâtre,
teintée du blanc des étamines. Épillets à une fleur.
Branches courtes et très fleuries de l’inflorescence.





%% =======================================================

\newpage

Aigremoine striée (ROSACÉES)  [344]
				Agrimonia striata\\

Ail des bois (LILIACÉES)  [660]					
				Allium tricoccum\\

Alpiste roseau (GRAMINÉES)  [804]
				Phalaris arundinacea\\

Amélanchier spp (ROSACÉES)  [315]				
				Amelanchier\\

Anaphale marguerite (COMPOSÉES)  [574]
				Anaphalis margaritacea\\

Ancolie du Canada (RENONCULACÉES)  [228]
				Aquilegia canadensis
Fleurs écarlates à éperons droits.\\


Ancolie vulgaire
				Aquilegia vulgaris
Fleurs bleues, pourpres ou blanches à éperons fortement recourbés.\\


Antennaire (COMPOSÉES)  [574]
				Antennaria howelii\\
				


Anthrisque des bois (persil sauvage) (OMBELLIFÈRES)  [420]
				Anthriscus sylvestris\\
				

Apocyn à feuilles d’Androsème (APOCYNACÉES)  [517]
				Apocynum androsaemifolium\\


Aralie à grappes (ARALIACÉES)  [412]
				Aralia racemosa\\


Aralie à tige nue
				Aralia nudicaulis\\


Arisème petit-prêcheur (ARACÉES)  [840]					
				Arisaema triphyllum (atrorubens)\\


Asaret du Canada (ARISTOLOCHIACÉES)  [219]			
				Asarum canadense\\			


Asclépiade commune (ASCLÉPIADACÉES)  [519]
				Asclepias syriaca\\
\newpage

Aster de la Nouvelle-Angleterre (COMPOSÉES)  [608]
				Aster (virgulus)  novae-angliae\\

Aster simple  [609]
				Aster simplex
				Symphyotrichum lanceolatum\\

Athyrie fausse-thelyptère (POLYPODIACÉES)  [132]
				Athyrium thelypteroides (-ridoides)
Limbe doublement divisé. Sores allongés (bâtonnets).\\

Athyrie fougère-femelle  [133]
				Athyrium Filix-femina\\

Aubépine spp (ROSACÉES)  [296-314]				( fruits
				Crataegus
Rameaux épineux.

Aulne crispé (BÉTULACÉES)  [151]
				Alnus crispa (viridis)
Rare au Qc (vulnérable). Cocottes persistantes sur longues queues.
Feuilles avec dents fines non doublement dentées.

Aulne rugueux
				Alnus rugosa (incana)
Cocottes persistantes, sans queue. ? A. viridis.
Feuilles épaisses, plissées, vert foncé, rugueuses, pubescentes sur les deux faces, doublement dentées.
\chapter*{B}

Barbarée vulgaire (CRUCIFÈRES)  [269]
				Barbarea vulgaris
Floraison : mai-juillet.

Bardane majeure (COMPOSÉES)  [570]
				Arctium Lappa
Hauteur : 1-2 m. Gros capitules sur un long pédoncule. La queue des feuilles du bas est pleine, ? A. minus.

Bardane mineure
				Arctium minus
Hauteur : 1-1,5 m. Feuilles teintées de blanc au-dessous. Petits capitules (1,5-3 cm).

Benoîte des ruisseaux (ROSACÉES)  [344]			(
				Geum rivale			rhizome (tisane)

Benoîte du Canada
				Geum canadense
Fleurs blanches, feuilles plus serrées, ? G. aleppicum : fleurs jaunes.


Benoîte d’Alep  [345]
				Geum aleppicum

Bermudienne commune (IRIDACÉES)  [669]
				Sisyrinchium montanum

Bident feuillu (COMPOSÉES)  [566]
				Bidens frondosa

Bleuet fausse myrtille (ÉRICACÉES)  [442]			(
				Vaccinium myrtilloides
Rameaux nettement velus, feuilles poilues sur les deux faces.

Bleuet (airelle) à feuille étroite					(
				Vaccinium angustifolium
Feuilles glabres.

Bouleau gris (à feuilles de peuplier) (BÉTULACÉES) [149]
				Betula populifolia
Feuilles triangulaires longuement acuminées.
Écorce ne s’exfoliant pas.
Taches noires sous les branches.

Bouleau blanc  [150]							( sève
				Betula papyrifera
Bouleau jaune (merisier)
				Betula alleghaniensis

Bouleau à feuilles cordées
				Betula cordifolia
    Betula papyrifera var. cordifolia.

Bourse à pasteur (CRUCIFÈRES)  [251]
				Capsella Bursa-pastoris

Brome inerme (GRAMINÉES)  [777]
				Bromus inermis

\chapter*{C}

Camomille des chiens (COMPOSÉES)  [592]
				Anthemis Cotula
Feuilles très divisées, comme les matricaires. ? C. Leucanthemum.

Campanule fausse-raiponce (CAMPANULACÉES)  [544]		(
				Campanula rapunculoides			racines

Cardamine carcajou (CRUCIFÈRES)  [258]				 (
				Cardamine (dentaria) diphylla	       rhizome
Carex stipité (CYPÉRACÉES)  [709]
				Carex stipita

Carex étoilé  [714]
				Carex echinata (angustior)

Carex à balais  [719]
				Carex scoparia

Carex plantain  [730]
				Carex plantaginea
Floraison très printanière

Carex crépu  [745]
				Carex crinita

Carex réfléchi  [754]
				Carex retrorsa

Carex gonflé  [756]
				Carex intumescens

Carotte sauvage (OMBELLIFÈRES)  [418]			( rhizome
				Daucus Carota
Centaurée noire (COMPOSÉES)  [568]
				Centaurea nigra
Bractées à sommet élargi, noir, finement frangé.

Centaurée jacée
				Centaurea Jacea

Centaurée des montagnes
				Centaurea montana

Céraiste vulgaire (CARYOPHYLLACÉES)  [208]
				Cerastium vulgatum
Fortement pubescent. ? S. graminea.

Cerisier de Pennsylvanie (ROSACÉES)  [320]		( fruits
				Prunus pensylvanica			sans noyau

Cerisier tardif  [321]
				Prunus serotina

Cerisier à grappes  [322]						( fruits
				Prunus virginiana			sans noyau
Petit arbuste (3-4 m), jusqu’à 9 m.

Chardon vulgaire (COMPOSÉES)  [582]
				Cirsium vulgare
Plante fortement épineuse. Capitules épineux (4-5 cm).

Chardon des champs  [583]
				Cirsium arvense
Petits capitules (2-2,5 cm) non épineux.

Chélidoine majeure (PAPAVÉRACÉES)  [248]
				Chelidonium majus

Chêne rouge (FAGACÉES)  [154]
				Quercus rubra
Feuilles lobées (7-9 lobes) à pointes acérées.

Chêne bicolore
				Quercus bicolor
Feuilles très pâles et pubescentes en-dessous. Lobes peu découpés à largeur maximale au-dessus du milieu.

Chêne blanc							( glands
				Quercus alba
Feuilles glabres en-dessous, à lobe terminal jamais grand. Lobes arrondis.


Chêne à gros fruits  [155]
				Quercus macrocarpa
Feuilles pubescentes en-dessous à lobe terminal souvent grand.

Chénopode blanc (CHÉNOPODIACÉES)  [194]		(
				Chenopodium album			graines
???: oie							gruau ou farine

Chèvrefeuille de Tartarie (CAPRIFOLIACÉES)  [537]
				Lonicera tatarica

Chèvrefeuille du Canada 
				Lonicera canadensis

Chicorée sauvage (COMPOSÉES)  [551]
				Cichorium Intybus

Chiendent (GRAMINÉES)  [788]
				Agropyron repens

Claytonie de Caroline (PORTULACACÉES)  [202]			(
				Claytonia caroliniana			bulbe
? C. virginica : feuilles étroites (8 fois plus longues que larges).
Rare au Qc, plus grégaire; colonies denses.
Clématite de Virginie (RENONCULACÉES)  [222]
				Clematis virginiana

Clintonie boréale (LILIACÉES)  [647]
				Clintonia borealis

Cohosh bleu (BERBÉRIDACÉES)  [237]
				Caulophyllum thalictroides

Coptide savoyane (RENONCULACÉES)  [230]
				Coptis trifolia

Cornouiller du Canada (CORNACÉES)  [407]			( fruits
				Cornus canadensis

Cornouiller stolonifère (Hart rouge)  [409]
				Cornus stolonifera

Cornouiller à feuilles alternes
				Cornus alternifolia

Cypripède rose (ORCHIDACÉES)  [820]
				Cypripedium acaule

\chapter*{D}

Dalibarde rampante (ROSACÉES)  [342]
				Dalibarda repens

Dennstaedtie à lobules ponctués (POLYPODIACÉES) [124]
				Dennstaedtia punctilobula
Limbe triplement divisé.

Dicentre capuchon jaune (FUMARIACÉES)  [245]
				Dicentra cucullaria

Dicentre capuchon rose
				Dicentra canadensis
Fleur rosée. Lobes courts et arrondis de la fleur. ? D. cucullaria : lobes étirés.

Digitaire sanguine (GRAMINÉES)  [808]
				Digitaria sanguinalis

Diplotaxis (drave) des murailles  (CRUCIFÈRES)
				Diplotaxis muralis



Dryoptéride du hêtre (moustaches) (POLYPODIACÉES)  [129]
				Phegopteris connectilis
				Dryopteris phegopteris

Dryoptère spinuleuse  [130]
				Dryopteris spinulosa (intermedia)
Limbe triplement divisé. Sores ronds.
? A. Filix-femina
Écailles brun pâle sur la crosse.
Spinule : petite pointe sur chaque dent du limbe.
Limbe vert plus foncé.

Dryoptère à sores marginaux
				Dryopteris marginalis
Limbe triplement divisé. Sores en forme de virgules.
Écailles brun noir sur la crosse.

\chapter*{E}

Égopode podagraire (OMBELLIFÈRES)  [425]
				Aegopodium Podagraria



Épervière florentine (COMPOSÉES)  [559]
				Hieracium florentinum
Très similaire à H. pratense. Feuilles lancéolées, peu ou pas poilues sur le dessus.

Épervière orangée
				Hieracium aurantiacum

Épervière piloselle  [560]
				Hieracium Pilosella
Un seul capitule. Fortement stolonifère (tapis).

Épervière des prés
				Hieracium pratense
Plusieurs capitules. Feuilles toutes basilaires et lancéolées, poilues sur les deux faces, ? H. florentinum.

Épervière vulgaire
				Hieracium vulgatum

Épiaire des marais (LABIÉES)  [499]				(
				Stachys palustris			tubercule

Épifage de Virginie (OROBANCHACÉES)  [488]
				Epifagus virginiana
Épilobe cilié (glanduleux) (ONAGRACÉES)  [372]
				Epilobium ciliatum
Fleurs à quatre pétales divisés en deux.

Épinette blanche (PINACÉES)  [143]
				Picea glauca
Pour toutes les épinettes, coupe transversale des aiguilles en losange. (
Rameaux pâles et glabres. Cônes cylindriques (3-6 cm).
Aiguilles vert bleuâtre.

Épinette noire
				Picea mariana
Rameaux rougeâtres et pubescents.
Cônes sphériques (2-3 cm) dont le bord des écailles est denté.

Épinette bleue (du Colorado)
				Picea pungens

Épinette rouge
				Picea rubens
Aiguilles recourbées en finale.
Cônes ovoïdes (3-5 cm) à écailles lisses, ? P. mariana.

Épinette de Norvège
				Picea abies
Épipacte petit-hellébore (ORCHIDACÉES)  [829]
				Epipactis Helleborine

Érable de Pennsylvanie (ACÉRACÉES)  [395]
				Acer pensylvanicum

Érable à épis
				Acer spicatum

Érable argenté  [396]
				Acer saccharinum
Feuilles simples lobées, à sinus profonds étroits et non dentés.
Grandes samares (40-70 mm) dont les ailes forment un angle de 90?.

Érable rouge							( samares
				Acer rubrum
Feuilles lobées, grossièrement dentées à sinus peu profonds larges et anguleux.
Petites samares (12-25 mm) dont l’angle des ailes est de 60?.

Érable à sucre  [398]
				Acer saccharum
Feuilles à sinus arrondis, non dentés.


Érigéron du Canada (COMPOSÉES)  [603] (Vergerette)
				Erigeron canadensis
Capitules très petits (2-4 mm). Rayons peu apparents.

Érigéron de Philadelphie  [604]
				Erigeron philadelphicus
Fleurs rosées.

Érigéron annuel
				Erigeron annuus
Fleurs blanches.

Érythrone d’Amérique (LILIACÉES)  [656]
				Erythronium americanum

Eupatoire maculée (COMPOSÉES)  [583]
				Eupatorium maculatum

Eupatoire perfoliée
				Eupatorium perfoliatum

Eupatoire rugueuse  [584]							?
				Eupatorium rugosum

\chapter*{F}

Fétuque rouge (GRAMINÉES)  [774]
				Festuca rubra

Fougère-aigle commune (POLYPODIACÉES) [125]
				Pteridium aquilinum

Fraisier des bois (ROSACÉES)  [342]
				Fragaria vesca (americana)
Se distingue de F. virginiana surtout par l’habitat : sites ombragés ou semi-ombragés, souvent rocheux. Graines à la surface de la fraise et non en dépressions.

Fraisier des champs
				Fragaria virginiana
Lieux ouverts et secs. Graines en surface mais dans une dépression.

Frêne blanc (OLÉACÉES)  [521]
				Fraxinus americana
7 à 9 folioles non sessiles.



Frêne rouge  [522]
				Fraxinus pennsylvanica
7 à 9 folioles non sessiles. Feuilles dentées du milieu à l’apex, au dessous duveteux.

Frêne noir
				Fraxinus nigra
7 à 11 folioles. Folioles sessiles sur le pétiole central.
Écorce comme du liège.

\chapter*{G}

Gadellier américain (SAXIFRAGACÉES)  [290]
				Ribes americanum
Fruits : baies noires sans piquant.

Gaillet piquant (RUBIACÉES)  [528]
				Galium asprellum

Galane glabre (SCROPHULARIACÉES)  [482]
				Chelone glabra

Galéopside à tige carrée (LABIÉES)  [499]
				Galeopsis Tetrahit
Galinsoga cilié (COMPOSÉES)  [585]
				Galinsoga ciliata

Gesse à larges feuilles (LÉGUMINEUSES)  [352]
				Lathyrus latifolius

Ginseng à trois folioles (ARALIACÉES)  [411]		(
				Panax trifolius				tubercule

Glécome lierre-terrestre (LABIÉES)  [495]
				Glecoma hederacea

Gnaphale des vases (COMPOSÉES)  [576]
				Gnaphalium uliginosum

Groseillier des chiens (SAXIFRAGACÉES)  [291]			(
				Ribes cynosbati					baies
Fruits : baies vertes (immatures) piquantes.
Présence d’épines aux nœuds (2-3) ? Gadelliers : sans épine.

Glycérie striée (GRAMINÉES)  [768]
				Glyceria striata


Gymnocarpe fougère du chêne (POLYPODIACÉES) [129]
				Gymnocarpium dryopteris
				Dryopteris disjuncta

\chapter*{H}

Hamamélis de Virginie (HAMAMÉLIDACÉES)  [218]
				Hamamelis virginiana

Hémérocalle fauve (LILIACÉES)  [654]
				Hemerocalis fulva

Hépatique noble (acutilobée) (RENONCULACÉES)  [229]
				Hepatica nobilis

Herbe à la puce (ANACARDIACÉES)  [392]			?
				Rhus toxicodendron
				Toxicodendron radicans

Herbe à poux (COMPOSÉES)  [562]  (Ambroisie)
				Ambrosia artemisiifolia

Hêtre à grandes feuilles (FAGACÉES)  [156]			( faînes
				Fagus grandifolia
Houx verticillé (AQUIFOLIACÉES)  [401]			?
				Ilex verticillata

Huperzie brillante (LYCOPODIACÉES)  [108]
				Huperzia lucidula
Fructification : sporanges jaunes à l’aisselle des feuilles.

\chapter*{I}

Impatiente du Cap (BALSAMINACÉES)  [399]
				Impatiens capensis

Iris versicolore (IRIDACÉES)  [667]				?
				Iris versicolor

\chapter*{J-K}

Julienne des dames (CRUCIFÈRES)  [256]
				Hesperis matronalis




\chapter*{L}

Laiteron des champs (COMPOSÉES)  [557]
				Sonchus arvensis
Capitules 3-5 cm. Tige 1-1,5 m.

Laiteron potager  [558]
				Sonchus oleraceus
Capitules 1-3 cm. Tige 0,5-1 m. Feuilles non épineuses.

Laiteron épineux
				Sonchus asper
Capitules plus petits (1,5-2,5 cm). Tige 0,5-1 m. Feuilles non découpées et fortement épineuses.

Léontodon automnal (COMPOSÉES)  [552]
				Leontodon autumnalis
Floraison : juin-octobre. Feuilles toutes basilaires semblables en forme à celles du pissenlit, mais plus petites. ? Épervières (feuilles lancéolées).

Lenticule mineure (Lentille d’eau) (LEMNACÉES)  [848]
				Lemna minor


Lépidie densiflore (CRUCIFÈRES)  [252]
				Lepidium densiflorum

Lilas vulgaire (OLÉACÉES)  [520]
				Syringa vulgaris

Linaire vulgaire (SCROPHULARIACÉES)  [470]
				Linaria vulgaris

Lis du Canada (jaune) (LILIACÉES)  [658]
				Lilium canadense

Liseron des champs (petit) (CONVOLVULACÉES)  [452]
				Convolvulus arvensis

Liseron des haies (grand)  [451]
				Convolvulus sepium
Présence de bractées. ? C. arvensis : sans bractée.

Lobélie gonflée (LOBÉLIACÉES)  [546] (Indian Tobacco)
				Lobelia inflata

Lotier corniculé (LÉGUMINEUSES)  [345]
				Lotus corniculatus
Lupuline (LÉGUMINEUSES)  [358]
				Medicago lupulina

Luzerne cultivée (LÉGUMINEUSES)  [358]
				Medicago sativa

Lysimaque ponctuée (PIRMULACÉES)  [430]
				Lysimachia punctata
Feuilles verticillées pubescentes.
Anneau marron distinctif sur la fleur.

Lysimaque Nummulaire  [431]
				Lysimachia Nummularia

Lysimaque terrestre
				Lysimachia terrestris

Lycopode claviforme (à massue) (LYCOPODIACÉES)  [109]
				Lycopodium clavatum
????? : loup
Plusieurs fructifications (strobiles) au bout d’une tige sans ramification.
Soie blanche qui coiffe l’extrémité de chaque ramification.

Lycopode foncé (dendroïde) 
				Lycopodium dendroideum (obscurum)
Lycopode innovant  [110]
				Lycopodium annotinum
Fructification: strobile terminal solitaire.
? L. clavatum.

Lycopode aplati
				Diphasiastrum complanatum

\chapter*{M}

Maïanthème du Canada (LILIACÉES)  [649]
				Maianthemum canadense

Marguerite (COMPOSÉES)  [589]
				Chrysanthemum Leucanthemum

Matricaire odorante (COMPOSÉES)  [590]
				Matricaria matricarioides

Matricaire maritime
				Matricaria maritima
Bractées sans poil et bordées d’une ligne brune près du capitule.
? Anthemis cotula : avec poil et odeur fétide.

Matteuccie fougère-à-l’autruche (POLYPODIACÉES)  [134]
				Matteuccia Struthiopteris
				Pteretis pensylvanica

Mauve musquée (MALVACÉES)  [380]
				Malva moschata

Médéole de Virginie (LILIACÉES)  [647]			( tubercule
				Medeola virginiana

Mélèze laricin (PINACÉES) [142]
				Larix laricina

Mélilot blanc (LÉGUMINEUSES)  [359]
				Melilotus alba

Mélilot officinal
				Melilotus officinalis
Inflorescence jaune.

Menthe à épis (LABIÉES)  [504]
				Mentha spicata
Inflorescence en épis terminal étroit.

Menthe poivrée  [505]
				Mentha piperita
Inflorescence en épis terminal large et carré. Les feuilles sont portées sur des tiges (au moins 4 mm), ? M. spicata.

Menthe du Canada
				Mentha canadensis
Verticilles floraux axillaires.

Mil (Phléole des prés) (GRAMINÉES)  [798]
				Phleum pratense

Millepertuis commun (HYPÉRICACÉES)  [284]
				Hypericum perforatum

Mitchella rampant (RUBIACÉES)  [524]
				Mitchella repens

Molène vulgaire (SCROPHULARIACÉES)  [469]
				Verbascum Thapsus

Monotrope uniflore (ÉRICACÉES)  [434]
				Monotropa uniflora

Morelle douce-amère (SOLANACÉES)  [464]
				Solanum Dulcamara

Moutarde des champs (CRUCIFÈRES)  [266]
				Brassica campestris
Feuilles rudes avec tache rouge à l’aisselle.

Myosotis laxiflore (BORAGINACÉES)  [458]
				Myosotis laxa

Myosotis à tiges dressées  [459]
				Myosotis stricta
Tige 10-20 cm, fleurs très petites.

Myrique baumier (MYRICACÉES)  [156]
				Myrica Gale

\chapter*{N}

Nénuphar jaune (petit) (NYMPHÉACÉES)  [239]
				Nuphar microphyllum

Nénuphar jaune (grand)  [240]				( racine
				Nuphar variegatum
Noisetier à long bec (BÉTULACÉES)  [152]		( noix
				Corylus cornuta

Noyer cendré (JUGLANDACÉES)  [158]			( noix
				Juglans cinerea

Nymphéa odorant (NYMPHÉACÉES) Water Lily [239]		(
				Nymphea odorata				racine

\chapter*{O}

Onagre bisannuelle (muriquée) (ONAGRACÉES)  [376]
				Oenothera biennis

Onoclée sensible (POLYPODIACÉES)  [133]
				Onoclea sensibilis

Orme d’Amérique (ULMACÉES)  [170]
				Ulmus americana
Feuilles rugueuses ? papier.

Orpin pourpre (CRASSULACÉES)  [286]		( feuilles/bulbe
				Sedum purpureum

Osmonde royale (OSMONDACÉES) [122]
				Osmunda regalis

Osmonde cannelle
				Osmunda cinnamomea

Osmonde de Clayton  [123]
				Osmunda Claytoniana

Ostryer de Virginie (BÉTULACÉES) [152]
				Ostrya virginiana

Oxalide des bois (OXALIDACÉES)  [384]		( feuilles
				Oxalis montana
Fleurs blanches. Présence d’acide oxalique comme R. acetosella.

Oxalide dressée							( feuilles
				Oxalis stricta

\chapter*{P}

Patience (rumex) crépue (POLYGONACÉES)  [190]
				Rumex crispus

Pâturin des prés (GRAMINÉES)  [771]
				Poa pratensis
Taille : 30 cm. Inflorescence verte teintée de rouge.
Feuilles basilaires pliées en 2 sur le long avec une nervure centrale rappelant une piste de ski.
Nombreux épillets contenant 3 à 5 fleurs.

Pervenche mineure (APOCYNACÉES)  [516]
				Vinca minor

Peuplier blanc (SALICACÉES)  [162]
				Populus alba

Peuplier baumier
				Populus balsamifera
Feuilles ovales finement dentées.

Peuplier à grandes dents
				Populus grandidentata

Peuplier faux-tremble  [163]
				Populus tremuloides
Petites feuilles réniformes sur longs pétioles.
Jeune arbre, écorce vert pâle.

Peuplier deltoïde
				Populus deltoides
Feuilles triangulaires grossièrement dentées.

Peuplier noir d’Italie (de Lombardie)  [164]
				Populus nigra

Phragmite commun (GRAMINÉES)  [765]
				Phragmites australis

Pigamon pubescent (RENONCULACÉES)  [234]
				Thalictrum pubescens

Pin blanc (PINACÉES)  [140]
				Pinus Strobus
Pour tous les pins, coupe transversale des aiguilles en triangle. (
Aiguilles fasciculées par 5.

Pin rouge  [141]
				Pinus resinosa
Aiguilles réunies par 2.

Pin gris/divariqué  [142]
				Pinus banksiana
Aiguilles courtes (2-4 cm), réunies par 2.
Pissenlit officinal (COMPOSÉES)  [553]
				Taraxacum officinale

Plantain majeur (PLANTAGINACÉES)  [509]
				Plantago major

Plantain lancéolé  [510]
				Plantago lanceolata

Polystic faux-acrostic (POLYPODIACÉES)  [128]
				Polystichum acrostichoides

Pommier (ROSACÉES)  [318]
				Malus

Populage des marais (RENONCULACÉES)  [222]
				Caltha palustris

Potentille frutescente (ROSACÉES)  [338]
				Potentilla fruticosa
Arbustive.



Potentille ansérine
				Potentilla Anserina
Feuilles basilaires composées de 9 à 31 folioles.

Potentille simple
				Potentilla simplex
Rhizome court ressemblant à un bulbe.
? P. reptans : rhizome noirâtre presque vertical.

Potentille de Norvège  [339]
				Potentilla norvegica
Tige dressée (30-80 cm), rigide, très poilue.

Potentille argentée  [340]
				Potentilla argentea

Potentille dressée
				Potentilla recta
Fleurs jaune-crème.

Prêle (fém.) faux-scirpe (ÉQUISÉTACÉES) [114]
				Equisetum scirpoides

Prêle panachée
				Equisetum variegatum
Prêle des champs
				Equisetum arvense
Pointes libres des gaines : brunes.

Prêle des prés
				Equisetum pratense
Pointes libres des gaines : blanches, d’où effet de roue de charriot.
Ramifications à l’horizontale, ? E. arvense : dressées.

Prêle des bois  [115]
				Equisetum sylvaticum
Pointes libres des gaines : rousses et très développées, à texture de papier.
Ramifications elles-mêmes ramifiées (rameaux divisés).

Pruche du Canada (PINACÉES)  [145]
				Tsuga canadensis

Prunelle vulgaire (LABIÉES)  [497]
				Prunella vulgaris

Pulmonaire officinale
				Pulmonaria officinalis

Pyrole elliptique (ÉRICACÉES)  [436]
				Pyrola elliptica
\chapter*{Q}

Quenouille à feuilles étroites (TYPHACÉES)  [854]		(
				Typha angustifolia

Quenouille à feuilles larges  [855]					(
				Typha latifolia

\chapter*{R}

Renoncule âcre (RENONCULACÉES)  [227]
				Ranunculus acris

Renouée à nœuds ciliés (POLYGONACÉES)  [181]
				Polygonum cilinode

Renouée liseron
				Polygonum Convolvulus

Renouée sagittée (gratte-cul)
				Polygonum sagittatum


Renouée de Pennsylvanie  [185]
				Polygonum pensylvanicum
Inflorescence rose.

Renouée persicaire  [187]
				Polygonum Persicaria
Inflorescence rose. Poils en bordure de la gaine, ? P. pensylvanicum

Renouée du Japon
				Polygonum cuspidatum

Rhinanthe crête-de-coq (SCROPHULARIACÉES)  [468]
				Rhinanthus Crista-galli

Ronce odorante (ROSACÉES)  [330]			( baies
				Rubus odoratus

Ronce pubescente  [331]					( baies
				Rubus pubescens

Ronce du mont Ida (framboisier)				( baies
				Rubus idaeus


Ronce alléghanienne (mûrier)  [334]				( baies
				Rubus allegheniensis

Rorippe sylvestre (CRUCIFÈRES)  [263]
				Rorippa sylvestris

Rosier  [324ss]
				Rosa

Rudbeckie hérissée (COMPOSÉES)  [593]
				Rudbeckia hirta

Rumex petit-oseille (POLYGONACÉES)  [188]		( feuilles
				Rumex Acetosella

\chapter*{S}

Salicaire pourpre (LYTHRACÉES)  [366]
				Lythrum Salicaria

Salsifis des prés (COMPOSÉES)  [552]
				Tragopogon pratensis


Sanguinaire du Canada (PAPAVÉRACÉES)  [248]
				Sanguinaria canadensis

Sapin baumier (PINACÉES) [146]
				Abies balsamea
Épines aplaties de forme ovale.

Saule (SALICACÉES)  [166ss]
				Salix

Sceau-de-Salomon pubescent (LILIACÉES)  [644]
				Polygonatum pubescens

Scirpe à gaines rouges (CYPÉRACÉES)  [696]
				Scirpus rubrotinctus

Scirpe noirâtre
				Scirpus atrovirens

Scirpe souchet  [697]
				Scirpus cyperinus

Scutellaire à fleurs latérales (LABIÉES)  [493]
				Scutellaria lateriflora
Séneçon vulgaire (COMPOSÉES)  [579]
				Senecio vulgaris

Silène enflée (CARYOPHYLLACÉES)  [205]
				Silene vulgaris (cucubalus)

Silène blanc (lychnis)
				Silene latifolia
Calice plus gros, fortement gonflé, à nervures rouge-vin.

Smilacine à grappes (LILIACÉES)  [650]			( fruits
				Smilacina racemosa

Sorbier d’Amérique (ROSACÉES)  [319]			( fruits
				Sorbus americana
Fruits : baies orangées (4-6 mm).

Sorbier décoratif
				Sorbus decora
Fruits : baies orangées (8-10 mm).

Souchet comestible (CYPÉRACÉES)  [684]			( bulbe
				Cyperus esculentus

Spirée tomenteuse (ROSACÉES)  [322]
				Spiraea tomentosa

Spirée à larges feuilles  [323]
				Spiraea latifolia

Stellaire à feuilles de graminée (CARYOPHYLLACÉES)  [210]
				Stellaria graminea
Feuilles glabres lancéolées.

Sureau blanc (CAPRIFOLIACÉES)  [530]			( fruits
				Sambucus canadensis
Fruit noir violacé à maturité (automne). Floraison estivale lorsque S. racemosa a déjà ses fruits.

Sureau rouge								?
				Sambucus racemosa
Floraison printanière. Fruits écarlates à maturité (été).

\chapter*{T}

Tabouret des champs (CRUCIFÈRES)  [252]
				Thlapsi arvense

Tanaisie vulgaire (COMPOSÉES)  [570]
				Tanacetum vulgare

Thélyptère de New-York (POLYPODIACÉES) [129]
				Thelypteris noveboracensis
Limbe doublement divisé. Sores ronds.

Tiarelle à feuilles cordées (SAXIFRAGACÉES)  [294]
				Tiarella cordiofolia

Tilleul d’Amérique (TILIACÉES)  [382]
				Tilia americana

Trèfle agraire (LÉGUMINEUSES)  [360]
				Trifolium agrarium
Fleurs jaunes.

Trèfle des champs (pied de lièvre)
				Trifolium arvense

Trèfle rouge  [361]
				Trifolium pratense


Trèfle blanc
				Trifolium repens

Trèfle Alsike (hybride)  [362]
				Trifolium hybridum

Trientale boréale (PIRMULACÉES)  [428]
				Trientalis borealis

Trille blanc (LILIACÉES)  [645]
				Trillium grandiflorum

Trille rouge  [646]
				Trillium erectum

Trille ondulé
				Trillium undulatum

Tussilage pas-d’âne (COMPOSÉES)  [594]
				Tussilago Farfara

Thuya occidental (CUPRESSACÉES) [140]
				Thuja occidentalis

\chapter*{U}

Uvulaire grande fleur (LILIACÉES)  [653]
				Uvularia grandiflora
Tige 50 cm. Fleurs (4 cm) jaune citron.

Uvulaire petite fleur  [654]
				Uvularia sessilifolia
Tige 20 cm. Fleurs (2 cm) jaune crème.

\chapter*{V}

Valériane officinale (VALÉRIANACÉES)  [538]
				Valeriana officinalis

Vélar fausse-giroflée (CRUCIFÈRES)  [270]
				Erysimum cheiranthoides

Vérâtre vert (LILIACÉES)  [662]						?
				Veratrum viride

Verge d’or à feuilles de graminées (COMPOSÉES)  [597]
				Solidago graminifolia
Verge d’or à tige zigzagante  [598]
				Solidago flexicaulis

Verge d’or rugueuse  [600]
				Solidago rugosa
Tiges et feuilles velues. Feuilles un peu plus large que S. canadensis.

Verge d’or du Canada  [601]
				Solidago canadensis
Velue sur les nervures inférieures de la feuille. Trois nervures caractéristiques.

Véronique à feuilles de thym (SCROPHULARIACÉES)  [473]
				Veronica serpyllifolia

Véronique de Perse  [474]
				Veronica persica

Véronique officinale
				Veronica officinalis

Véronique en écusson  [475]
				Veronica scutellata


Verveine hastée (VERBÉNACÉES)  [490]
				Verbena hastata

Vesce jargeau (LÉGUMINEUSES)  [349]
				Vicia Cracca

Vigne des rivages (VITACÉES)  [405]				( fruits
				Vitis raparia

Vigne vierge  [406]							?
				Parthenocissus quinquefolia

Vinaigrier (ANACARDIACÉES)  [391]
				Rhus typhina

Violette pâle (VIOLACÉES)  [278]
				Viola macloskeyi (pallens)

Violette du Labrador  [280]
				Viola labradorica (conspersa)

Viorne bois-d’orignal (CAPRIFOLIACÉES)  [533]
				Viburnum lantanoides
				Viburnum alnifolium
Viorne trilobée							( fruits
				Viburnum trilobum

Viorne de Rafinesque  [534]
				Viburnum Rafinesquianum

\chapter*{W-X-Y-Z}


	
\end{document}	
