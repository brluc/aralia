\documentclass[book,10pt,a4paper,onecolumn,openany]{memoir}
\setlrmarginsandblock{2.5in}{2.5in}{*}
\setulmarginsandblock{3.5in}{3.5in}{*}
\fixthelayout
%%\raggedbottom

%%\chapterstyle{demo}
%%\chapterstyle{demo2}
\chapterstyle{bianchi}
%%\chapterstyle{chappell}
%%\chapterstyle{lyhne}
%%\chapterstyle{southall}
%%\chapterstyle{veelo}

\usepackage[utf8]{inputenc}
\usepackage[LGR,T1]{fontenc} %% LGR for Greek.
\usepackage[sc]{mathpazo}
\usepackage[french]{babel}
\usepackage{hyphenat}

%% Use this instead of \emph for Latin text.
\newcommand{\latin}[1]{%
  {\itshape #1}}

\newcommand{\gk}[1]{%
  \begingroup\fontencoding{LGR}\selectfont#1\endgroup}

%% Greek accents with breathing.
\DeclareTextAccentDefault{\accdialytika}{LGR}
\DeclareTextAccentDefault{\acctonos}{LGR}
\DeclareTextAccentDefault{\accdasia}{LGR}
\DeclareTextAccentDefault{\accpsili}{LGR}
\DeclareTextAccentDefault{\accvaria}{LGR}
\DeclareTextAccentDefault{\accperispomeni}{LGR}
\DeclareTextAccentDefault{\accdialytikaperispomeni}{LGR}
\DeclareTextAccentDefault{\accdialytikatonos}{LGR}
\DeclareTextAccentDefault{\accdasiaperispomeni}{LGR}
\DeclareTextAccentDefault{\accdasiavaria}{LGR}
\DeclareTextAccentDefault{\accdasiaoxia}{LGR}
\DeclareTextAccentDefault{\accpsiliperispomeni}{LGR}
\DeclareTextAccentDefault{\accpsilioxia}{LGR}
\DeclareTextAccentDefault{\accpsilivaria}{LGR}
\DeclareTextAccentDefault{\accinvertedbrevebelow}{LGR}
\DeclareTextAccentDefault{\accbrevebelow}{LGR}

\pretitle{{\vskip 2cm} \centering\huge}
\posttitle{\par\vskip 1.25cm}
\preauthor{\centering\large}
\postauthor{\par}
\predate{\par\centering\vskip 0.5cm}
\postdate{\par\vskip 1cm\vfill}

\title{%
  {\bfseries Les Noms de Nos Plantes}\\[.25em]
  {\large Répertoire progressif d'identification}\\[.25em]
  {\Large 5e edition}}
\author{{\Large\scshape SBL Bushcraft Project}}
\date{Mai 2020}

\newenvironment{nonewpage}
  {\par\nobreak\vfil\penalty0\vfilneg
   \vtop\bgroup}
  {\par\xdef\tpd{\the\prevdepth}\egroup
   \prevdepth=\tpd}

\usepackage{lipsum}

%% #1 => name
%% #2 => latin name
%% #3 => family
%% #4 => reference page
\newcommand{\herbe}[4]{
  \noindent\textbf{\large #1}\hfill {[#4]}
  \par\noindent\hfill {\scshape #3}
  \par\noindent\hfill {\itshape #2}
  \par\smallskip
}

%% #1 => name
%% #2 => latin name
%% #3 => other latin name
%% #4 => family
%% #5 => reference page
\newcommand{\herbex}[5]{
  \noindent\textbf{\large #1}\hfill {[#5]}
  \par\noindent\hfill {\scshape #4}
  \par\hfill {\itshape #2}\hskip2pt /\hskip2pt {\itshape #3}
  \par\smallskip
}

\newcommand{\herbskip}{\vskip 1cm}

%% Example of hyphenation rule.
\hyphenation{éc-orce}
\hyphenation{confon-dre}
\hyphenation{rou-ges}
\hyphenation{bran-ches}
\hyphenation{Lo-bes}
\hyphenation{lancéo-lées}
\hyphenation{sphéri-que}
\hyphenation{ha-bitat}
\hyphenation{pré-sence}
\hyphenation{épineu-ses}
\hyphenation{jeu-ne}
\hyphenation{natura-lisées}